\documentclass[12pt,a4paper]{article}
\usepackage[left=2cm,right=2cm,top=2.5cm,bottom=2.5cm]{geometry}
\usepackage[utf8]{inputenc}
\usepackage[T1]{fontenc}
\usepackage{multirow}

\usepackage[english]{babel}
\usepackage{cite}

\usepackage{makeidx}
\makeindex

\usepackage{graphicx}
\graphicspath{{Pictures/}} 

\usepackage{amsmath}
\usepackage{amssymb}

\input{thm_header}


\title{High-Assurance Quasi-Synchronous Systems}
\date{}
\author{Robin Larrieu -- \'Ecole Polytechnique \\ Natarajan Shankar -- SRI International}

\begin{document}

%\newgeometry{left=3cm, right=3cm, top=4cm, bottom=2.5cm} % different margins for abstact + TOC
\maketitle

\begin{abstract}
The design of a complex cyber-physical system is centered around one
or more models of computation (MoCs).  These models define the
semantic framework within which a network of sensors, controllers, and
actuators operate and interact with each other.  In this paper, we
examine the foundations of a quasi-synchronous model of computation
Our version of the quasi-synchronous model is inspired by the Robot
Operating System (ROS).  It consists of nodes that encapsulate
computation and topic channels that are used for communicating between
nodes.  The nodes execute with a fixed period with possible jitter due
to local clock drift and scheduling uncertainties, but are not
otherwise synchronized.  The channels are implemented through a
mailbox semantics.  In each execution step, a node reads its incoming
mailboxes, applies a next-step operation to update its local state,
and writes to all its outgoing mailboxes.  The underlying transport
mechanism implements the mailbox-to-mailbox data transfer with some
bounded latency.  Messages can be lost if a mailbox is over-written
before it is read.  We prove a number of basic theorems that are
useful for designing robust high-assurance cyber-physical systems
using this simple model of computation.  We show that depending on the
relative rates of the sender and receiver, there is a bound on the
number of consecutive messages that can be lost.  By increasing the
mailbox queue size to a given bound, message loss can be eliminated.
We demonstrate that there is a bounded latency between the time a
message is sent and the time that it is processed, which in turn can
be used to bound the end-to-end sense-control-actuate latency.  We
illustrate how these theorems are useful in designing and verifying a
correct thermostat-based heating system.  Our proofs have been
mechanically verified using the Prototype Verification System (PVS). 
\end{abstract}
\newpage
 
%\renewcommand{\contentsname}{Table of Contents \vspace{1cm}}
\tableofcontents

\restoregeometry % return to default margins
\newpage

\section{Introduction}
The DARPA-funded project \emph{High Assurance Cyber-Military System (HACMS)} aims to produce systems and software with proved reliability and security.
Systems designed at SRI as part of this project are inspired by the ROS (Robot Operating System) architecture.
In ROS, we use the abstraction of \textit{nodes} and \textit{topics} to describe the different components and communication channels. 
At initialization, each node is declared subscriber to some topics and publisher on others. At each step, a node reads messages from the subscribed topics, execute a computation and publish on the topics it is in charge.

Each node is set to execute at a given frequency, but the actual rate to publish messages may vary because of clock bias, or a computing duration depending on the inputs. Since several nodes may execute physically on the same processor, scheduling randomness adds to this imprecision. 
When a message is published on a topic, it may need a certain delay before it is received by the subscribers. This delay is caused by the nature of the communication link (network/shared memory) and the underlying mailbox system.
For these reasons, we cannot ensure a subscriber will have new messages at each step, or that sent messages will be processed by the subscriber (they may be overridden before that).
Figure~\ref{msg} gives examples for these problems.

\begin{figure}[h]
\begin{center}
\includegraphics[height=6cm]{messages.jpg}
\caption{Uncertainty for the messaging system}\label{msg}
\end{center}
\end{figure}


In this report, we will establish a worst case scenario for this messaging system. For example, knowing that a node will always have available at least one message with a bounded age can be sufficient to assert a control loop maintains a state precisely enough around the desired value.
This can also be used for monitoring purposes: if a node detects a behavior outside this worst case scenario, it can raise an alert to a supervisor. By collecting these alerts, the supervisor could detect a compromised/crashed node, or a broken link, and decide how to fix the problem.

In Section 3, we give more details about the ROS architecture and the model used to represent it. In Section 4, we prove low-level properties for the messaging system. In sections 5 to 7, we give assurance claims for simple end-to-end properties. Section 5 proves that a basic control loop maintains a state around a set value. Sections 6 and 7 give two different models for obstacle avoidance.
In Section 2, we give more details about the ROS architecture and the model used to represent it. In Section 3, we prove low-level properties for the messaging system. In Section 4, we give assurance claims for a simple end-to-end property that a basic control loop maintains a state around a set value.
\newpage

\section{The ROS architecture}
\subsection{Overview of ROS functioning}

The aim here is not to give a complete description of ROS (the interested reader could refer to ROS manuals, for example \cite{O'Kane201310}). However, a few notions about ROS abstraction could be needed to understand following sections.

The main purpose of ROS is to provide a framework for distributed computing. Rather than writing one complex program that runs the whole robot, designers prefer writing several simpler programs (potentially running on different computers) in charge of a specific task: 
one program that uses sensors to scan the environment, one responsible to interact with the operator, another one controlling the wheels and so on

Each of these programs is represented by a ROS \textit{node} that can send messages to other nodes.
\textit{Topics} are declared to group messages with the same function, and each node can be publisher or subscriber to one or more topics. 
The ROS master gives information to publishers (for example IP addresses or location of shared buffers), so when publishing a message on a topic, they can send messages to all subscribers to this topic.
Messages received by a subscriber are stored in a queue until they are computed. If the queue is full when a new message is received, the oldest message is thrown away. 



\subsection{Modeling assumptions}\label{ROSarch} 

\paragraph{Topic assumptions:}
To avoid jamming, we require at most one publisher per topic\footnote{$ $ no publisher on a topic generates a warning, but is considered non-critical}; this is secured in HACMS by message authentication and firewall rules.
For simplification purpose, we assume in this report that there is only one subscriber on each topic.
This assumption does not lead to a loss of generality: a topic with $n$ subscribers could be seen as $n$ topics with one subscriber.
Finally, we assume that the transmission delay for a given message is bounded.

Then, a topic can be characterized by two nodes (the publisher and the subscriber), the maximum transmission delay and the queue size for the subscriber.


\paragraph{Node assumptions:}
Nodes are discrete simple tasks executed regularly.
Each node is supposed to execute at a given frequency, but the actual rate may vary.
The execution could be non-periodic, but we assume that the time between two consecutive steps (or pseudo-period) is inside some known interval $[minT, maxT]$ with $minT > 0$.
The typical example is when the frequency is known with a certain drift $\rho$, and the actual instantaneous frequency is in $[(1-\rho).f, (1+\rho).f]$.

We also do the simplifying assumption that the task at each step is executed instantaneously. 
Actually, in the case of a node that is only a publisher (resp. subscriber)\footnote{$ $ for example a sensor (resp. actuator)}, the time of the execution event can be taken as the time when the node sends (resp. reads) the message, which are atomic operations.
In the case of an intermediate node, that is both subscriber and publisher on different topics, the computing duration between these two operations could just be added to the delay of the published topic.

Then a node can be characterized by the upper and lower bound for the pseudo-period between consecutive steps.


\paragraph{Notations:}
Events are described by functions from $\mathbb{N}$ (the system runs forever) to $\mathbb{R^+}$ that give the time when this event happens. For example, if $e$ is the function used to describe the executions of some node, $e(n)$ is the time when the nth execution occurs.

\begin{defin}\Thmname{Node execution}

Each node $N$ is defined by its minimal pseudo-period $minT(N)$ and maximal pseudo-period $maxT(N)$. 
Then an execution of $N$ is any function $e$ such that
\[ e(0) = 0 \quad \textrm{\emph{and}} \quad \forall n \in \mathbb{N}, \, 
e(n) + minT(N) \leq e(n+1) \leq e(n) + maxT(N) \]
\end{defin}

The assumption $e(0) = 0$ could seem too strong because it implies all nodes start exactly at the same time (which is quite unrealistic). First, having a common origin simplifies inductions, that are widely used in proofs. 
Second, we could prove that for any initial time shift $\Delta$, with an interval $[minT(N), maxT(N)]$ as small as desired, there exist executions for $N$ and $N'$ such that \mbox{$e(n) - e'(m) = \Delta$} for some $n$ and $m$.
This means that our model can be generalized to allow a time shift by assuming nodes $N$ and $N'$ send their default value (which is supposed to lead to a safe behavior) before periodic step $n$ and $m$.
An induction gives the result for more than 2 nodes:

The two nodes $N$ and $N'$ can be grouped into a single node $\hat{N}$ with period a common multiple of admissible periods for $N$ and $N'$: $\hat e(k) = k.T = e(n+ka) - e(n) = e'(m+kb) - e'(m)$.
Assume $N''$ has an initial time shift of $\Delta '$ with $N$. Apply previous result with $\hat N$ and $N''$ for a time shift of $\Delta ' - e(n)$: $e''(l) - \hat e(k) = \Delta ' - e(n)$. Then, $e''(l) - e(n+ka) = \Delta '$ and $e''(l) - e'(m+kb) = \Delta' - \Delta$ (initial time shift between $N'$ and $N''$).

\begin{defin}\Thmname{Message reception}\label{reception}

Let $T$ be a topic with a maximum transmission delay of $D$, and let $s$ be an execution of its publisher $S$ (associated to the event \emph{a message is sent}). 
A reception of these messages is any function $r$ such that
\[ \forall n \in \mathbb{N}, \, s(n) \leq r(n) \leq s(n) + D \quad 
\textrm{\emph{and}$\quad r$ is injective} \]
\end{defin}

The hypothesis that $r$ is injective means that two different messages cannot be received exactly at the same time. This is linked to hardware limitations, and is important to represent the queue: if messages $m$ and $n$ are received exactly at the same time, which one comes before the other in the queue?   

Note that this definition does not assume the channel conserves the order of messages. We could have $r(k) > r(k+1)$ and therefore, the message received just after message $n$ is not necessarily message $n+1$.
However, we can define a function $next$ such that there is no message received between $r(n)$ and $r(next(n))$.
Then the iterated function $Nth\_next$ is such that there is exactly $N-1$ messages received between $r(n)$ and $r(Nth\_next(N, n))$.
This means $next(n)$ is the message received just after $n$ is, and $Nth\_next(N, n)$ is the N'th message received after $n$ is.

\vbox{
\begin{defin}\Thmname{Processed message}

Let $T$ a topic with a subscribers queue size of $L$. Let $s$ and $c$ be an execution of its publisher and subscriber and $r$ be a reception of sent messages.

By definition, we have $\big |\{k\, | \, r(n) \leq r(k) < r(Nth\_next(L, n))\} \big | = L$. Then, message $n$ is in the queue at time $t$ iff \footnote{$ $ Actually, depending on the implementation, the message could be removed from the queue during a computation, but before $r(Nth\_next(L, n))$.
This does not affect the definition.}
 \mbox{$r(n) \leq t < r(Nth\_next(L, n))$}
 
A message $n$ is said to be \emph{processed} if and only if 
\[ \exists k \in \mathbb{N}, \quad r(n) \leq c(k) < r(Nth\_next(L, n)) \]
\end{defin}
}

The message is said to be \emph{lost} or \emph{dropped} otherwise



\section{Low-level messaging system}
In this section, we consider a topic $T$, characterized by a publisher $P$, a subscriber $S$, a maximum transmission delay $D$, and a queue length $L$.

For $n \in \mathbb{N}$, we note $s(n)$ and $r(n)$ respectively the time when the n'th message is sent and received. We also note $c(n)$ the time when the subscriber executes its n'th computation.

\subsection{Latency}

For a processed message, we define its latency as the duration between the time it is sent and the time it is computed for the first time (see Figure~\ref{latency}):

\begin{figure}[h]
\begin{center}
\includegraphics[height=5cm]{latency.jpg}
\caption{Latency definition}\label{latency}
\end{center}
\end{figure}

\begin{defin} \Thmname{Latency}
\[ Latency(n) = c \Big ( min(\{k \in \mathbb{N} | r(n) \leq c(k)\}) \Big ) - s(n) \]
\end{defin}

Since $\forall k, c(k + 1) \leq c(k) + maxT(S)$, we have
$c \big ( min(\{k \in \mathbb{N} | r(n) \leq c(k)\}) \big ) - r(n) \leq maxT(S)$. Therefore,
\begin{thm}\Thmname{Latency bound}
\[ Latency(n) \leq maxT(S) + D \]
\end{thm}

The latency is only defined for a processed message, so this gives little information for assurance properties. However, it can be used to detect a clock failure or a certain type of attack: 
assume for example that the subscriber reads a new message, but the delay since the messages timestamps exceed the latency bound. 
It could be that the clock of one of the nodes drifted significantly from real time, or it may mean that the message had been recorded and was resent later by an attacker.



\subsection{Overtaking}

The model defined in section \ref{ROSarch}, allows an older message to be delivered after a more recent one. We may want to avoid this even if the mailbox system alone doesn't assert that this property holds.
We give here a simple condition that asserts that messages are received in the order they are sent. 

Avoiding overtaking can yield better bounds in some cases to assert other properties are satisfied. By choosing good nodes parameters, a developer can also secure the order of received messages and rely on this to prove correctness of an application.

We have $r(n) \leq s(n) + D$, $s(n) + minT(P) \leq s(n + 1)$ and $s(n + 1) \leq r(n + 1)$. Hence the following theorem:

\begin{thm} \Thmname{Overtaking}
\[ D < minT(P) \quad \implies \quad \forall n \in \mathbb N, \quad r(n) < r(n + 1)\]
\end{thm}


\subsection{Number of lost messages}

As we saw before, messages can be overwritten in the queue before they are actually computed by the subscriber, which means these messages are never processed. However, we can ensure the subscriber does not miss too many consecutive messages

\subsubsection[Consecutive lost messages]{Upper bound for the number of consecutive lost messages}

\begin{defin}\label{consec}
\Thmname{Consecutive lost messages}\\
Formally, we can define the property "the subscriber never misses N consecutive messages" by
\[ \forall k \in \mathbb{N}, \exists l < N, \quad \textrm{\emph{message $k + l$ is processed}}   \]
\end{defin}

First, we have this fundamental lemma:
\begin{lem}\label{existsProcess}
If $c(n) \geq t + D + maxT(P)$, there exists a message $k$ with $t < r(k) \leq c(n)$ that is processed.
\end{lem}

\begin{proof}
$r(0) \leq D < c(n)$. Let $m$ be the maximum of the $l \in \mathbb{N}$ such that $r(l) \leq c(n)$. Since $r(m + 1) > c(n)$, we have $t < r(m) \leq c(n)$.

The set $S = \{l \in \mathbb{N}, t < r(l) \leq c(n)\}$ is finite and nonempty. There exists a $k \in S$ such that $\forall l \in S, r(l) \leq r(k)$. By construction, such a $k$ solves the problem because no message is received between $r(k)$ and the next execution of the subscriber.
\end{proof}

Basic inequality reasoning give this result:

\begin{lem}\label{order_received}
\[ r(m) > s(n) + D - minT(P) \implies n \leq m \]
\[ r(m) < s(n) + minT(P) \implies m \leq n\]
\end{lem}

\begin{thm}\label{clm}\Thmname{Consecutive lost messages}

If $N.minT(P) > 2.D + maxT(S) + maxT(P) - minT(P)$, then the subscriber never misses $N$ consecutive messages.
\end{thm}

\begin{proof}
According to definition~\ref{consec}, given $k \in \mathbb N$, we prove there exists an $l < N$ such that message $k + l$ is processed.

The subscriber executes at least once in every interval of time of length $maxT(S)$. In particular, 
\[ \exists n \in \mathbb N, \quad 
\left\{
\begin{array}{l}
s(k) + 2.D + maxT(P) - minT(P) < c(n) \\
c(n) \leq s(k) + 2.D + maxT(P) - minT(P) + maxT(S)
\end{array} \right. \]

Lemma \ref{existsProcess} gives $\exists l \in \mathbb{N}, s(k) + D - minT(P) < r(l) \leq c(n)$ such that the message $l$ is processed. Since $s(k + N - 1) \geq (N-1) . minT(P)$, with the given condition on $N$, $c(n) < s(k + N - 1) + minT(P)$.

With lemma \ref{order_received}, we have $k \leq l \leq k + N - 1$, and $l$ is processed by construction.
\end{proof}

\paragraph{Case without overtaking :} Assume now that $\forall k \in \mathbb N,\,\, r(k) < r(k + 1)$. In that case, with $m = max({k \in nat, r(k) \leq c(n)})$ we have $r(k) \leq c(n) \implies r(k) \leq r(m)$ which ensures message $m$ is processed.

With $N.minT(P) > D + maxT(S)$, we have $\exists n \in \mathbb N, s(k) + D \leq c(n) < s(k + N)$, then $r(k) \leq c(n) < r(k + N)$. We deduce message $m$ is processed with $k \leq m < k + N$.

\begin{thm}\label{clm2}\Thmname{Consecutive lost messages -- no overtaking}

Assume $N.minT(P) > D + maxT(S)$ and $\forall k \in \mathbb{N},\,\, r(k) < r(k + 1)$. Then the subscriber never misses $N$ consecutive messages.
\end{thm}


\subsubsection{Influence of the queue length}

As one can expect, increasing the message queue size leads to a lower message loss rate. For more comprehension, we first prove this in the case without overtaking. 
Then, we give an idea of the proof in the general case.

Assume we have $\forall k \in \mathbb{N},\,\, r(k) < r(k + 1)$ and we never drop $N$ ($N > 1$) messages with a queue length of $L$. Given $k \in \mathbb N$, there exists $l < N$ such that message $k + l$ is processed. 
If $l < N - 1$, the result is proved. If $l = N - 1$, it means the message $k + N - 1$ was in the queue when some computation $c(n)$ occurred.
With a queue length of $L + 1$, the message $k + N - 2$ was in the queue when $c(n)$ occurred, which means it was processed.

\paragraph{ }
In the general case, we assume the property "never miss $N$ consecutive messages" holds whatever $r$ is as soon as the constraints with $s$ and $D$ are respected (theorem \ref{clm} gives this assurance).
Like in the previous proof, we assume $k + N - 1$ is processed and none of the $k + l$ with $l < N - 1$ is. 
Among non-processed messages received before $r(k + N - 1)$, let $m$ be the one received last. With a queue length of $L + 1$, $m$ is processed with the same argument as before. The difficult part is to prove that this $m$ exists and must be one of the $k + l$, $l < N - 1$.
For this, we construct an other $r'$, consistent with the constraints on $s$ and $D$ but with $N$ consecutive messages lost (which is a contradiction).

Assume that none of the $k+l$ verifies $r(k+l) < r(k+N -1)$. Let $r'$ be a new reception sequence where reception of message $k+N-1$ is delayed to $r(k) + \varepsilon$ (see Figure~\ref{contr}.a),
where $\varepsilon$ is small enough to ensure $r'$ verifies constraints of definition~\ref{reception} and that no message is received between $r'(k)$ and $r'(k+N-1)$.
In this case, messages $k$ to $k+N-2$ are lost because messages that overwrote them with reception $r$ still do with reception $r'$, and message $k+N-1$ is lost because it is received between message $k$ (that is lost) and the next processed message.
This proves $m$ exists and verifies $r(k+l) \leq r(m) < r(k+N-1)$ for some $l$.

Assume $m > k+N-1$. Let $r'$ be the reception sequence where reception of messages $m$ and $k+N-1$ are switched (no change for other messages; see Figure~\ref{contr}.b). $m$ was lost with $r$ means $k+N-1$ is lost with $r'$. Again, messages $k$ to $k+N-1$ are lost with $r'$.

Assume $m < k-1$ (case $m=k-1$ is impossible: $k-1$ must be processed since none of the $k,\dots,k+N-2$ is). If $r(k-1) < r(m)$, switching reception of $k-1$ and $m$ (see Figure~\ref{contr}.c) gives a contradiction with the same argument as previously.
Otherwise, we consider the reception sequence $r'$ where reception of message $k-1$ is anticipated to $r(m)-\varepsilon$. Then, messages $k-1$ to $k+N-2$ are lost (see Figure~\ref{contr}.d).

\begin{figure}[h]
\begin{center}
a) Case $\forall l \in [0, N-2], r(k + l) > r(k+N-1)$ (includes the case where $m$ does not exist)\\
\includegraphics[width=16cm]{contr1.jpg}\\
b) Case $m > k+N-1$\\
\includegraphics[width=16cm]{contr2.jpg}\\
c) Case $m < k-1$ and $r(m) > r(k-1)$\\
\includegraphics[width=16cm]{contr3.jpg}\\
d) Case $m < k-1$ and $r(m) < r(k-1)$\\
\includegraphics[width=16cm]{contr4.jpg}\\
(green: processed -- red: lost -- blue: status unknown/irrelevant)
\caption{Getting contradictions}\label{contr}
\end{center}
\end{figure}

\paragraph{ }By a simple induction, we then get the following result:

\begin{thm} 
Let $N$ satisfies conditions of either theorem~\ref{clm} or theorem~\ref{clm2}. Let $m < N$ and assume $L > m$. Then the subscriber never miss $N - m$ consecutive messages.

In particular, if $L \geq N$, no message is lost.
\end{thm} 

\subsection{Age of processed messages}

In this section, we prove that at each computation, the subscriber gets (from a newly received message or with a backup from previous computation) a reasonably recent message.

This is essentially a bound on the age of processed messages that will be used in next sections to prove end-to-end properties. It is also helpful to detect errors: when the latest available message to the subscriber is older than the bound, it can raise a timeout flag. By collecting these flags, a monitor could guess whether the publisher node crashed or there is a network failure.

\subsubsection{Definition and basic properties}

The aim is to model the following behavior (see Figure~\ref{age}):
 At each computation, if messages are available in the queue, the node chooses the most recent one and saves it to give a backup solution for next computation. 
If not, it uses the backup given by previous step (while no message has been received, and no backup is available, a default value -- set at initialization time -- is used)

\begin{figure}[h]
\begin{center}
\includegraphics[height=5cm]{age.jpg}
\caption{Age definition}\label{age}
\end{center}
\end{figure}


\vbox{
\begin{defin}\label{ageDef}\Thmname{Age for the most recent available message}

For $n \neq 0$, we define the (finite, but maybe empty) set of messages computed at step n by
\[PL(n) = \big \{k \in \mathbb{N}, c(n - 1) < r(k) \leq c(n) 
\wedge \textrm{\emph{message $k$ is processed}} \big \}\]

Then, the age of the most recent available message can be defined by the recursive function \\
\emph{\texttt{Age($n$) = \{\\
\indent if $n=0$ then 0 \\
\indent else \{if $PL(n) = \emptyset$\\
\indent\indent then \{Age($n - 1$)$ + c(n) - c(n-1)$\}\\
\indent\indent else \{$c(n) - s(max(PL(n))$\}\\
\}\}
}}
\end{defin}
}

\vbox{
\begin{lem}\label{ageLem}
A simple induction over $n$ gives \texttt{\emph{Age}($n$)}$\leq c(n)$. 

Also, with the same argument as in the proof of lemma~\ref{existsProcess}, we have:
\[PL(n) = \emptyset \,\, \Longleftrightarrow \,\, 
\{k \in \mathbb{N}, c(n - 1) < r(k) \leq c(n)\} = \emptyset
\]
\end{lem}
}

\subsubsection{Upper bound in worst case scenario}

\begin{thm}\label{maxAge}\Thmname{Age bound}
\[ \texttt{\emph{Age($n$)}} < 2.D + maxT(P) \]
\end{thm}

\begin{proof}
First, we see by a simple induction over n that:
\[r(k) \leq c(n) \implies \texttt{Age($n$)} \leq c(n) - r(k) + D\]
(if $PL(n) = \emptyset$, lemma~\ref{ageLem} gives $r(k) \leq c(n-1)$. Otherwise, using the definition of a processed message, we either have $r(k) \leq r(max(PL(n)))$ or $k \leq max(PL(n))$. In both cases, \mbox{$s(max(PL(n))) \geq r(k) - D$})

In the case $c(n) \leq D + maxT(P)$, lemma~\ref{ageLem} gives the result. Otherwise, by lemma~\ref{existsProcess}, there exists a message $k$ such that $c(n) - D - maxT(P) < r(k) \leq c(n)$, which proves the theorem with the property above.
\end{proof} 

When messages are delivered in the same order they were sent, we get a better bound:

\begin{thm}\label{maxAge2}\Thmname{Age bound -- no overtaking}
\[ \forall k \in \mathbb{N},\,\, r(k) < r(k+1) \quad \implies \quad
\texttt{\emph{Age($n$)}} < D + maxT(P) \]
\end{thm}

\begin{proof}
When no overtaking is possible, the message $m(n) = max(\{k\in \mathbb{N},\, r(k) \leq c(n)\})$ (assuming the considered set is non empty) is processed. 

Since $r(m(n)+1) \leq s(m(n)) + D + maxT(P)$, we have 
\[ c(n) - D -maxT(P) < s(m(n)) \leq c(n) \]

$r(m(n)) \leq c(n-1) \implies m(n) = m(n-1)$. Therefore, we get by induction 
\[ \texttt{Age($n$)} = 
\left\{ \begin{array}{ll}
c(n) - m(n) &\textrm{ if $ r(0) \leq c(n)$}\\
c(n) &\textrm{ if $ r(0) > c(n)$, which means $c(n) < D$}
\end{array}\right.
\]
\end{proof}





\section{Assurance claim for the plant controller}
\input{plant_controller}

%\section{Further work: obstacle avoidance}
%There exist proved obstacle avoidance procedures based on a \emph{dynamic window} algorithm (see, e.g. \cite{Mitsch-RSS-13}).
Among reachable speeds from the current state within some small delay (the \emph{dynamic window}), the system choses one that is both optimized in the sense of an objective function, and allows the vehicle to stop before hitting an obstacle.
In this section, we apply this strategy to our system and we prove that it effectively allows to stop soon enough.

\subsection{Objectives}

We consider the worst possible case that the vehicle runs in straight line directly directed to a fixed obstacle. We assume a sensor gives distance to the obstacle and vehicle speed up to a certain accuracy.
When the measured distance is too low, the controller sends an \emph{emergency stop} command that should bring the vehicle to full stop without hitting the obstacle.

The mechanical properties of the system give a maximum acceleration $A$ and a braking power $b$; such that $-b \leq \dot v \leq A$ and, while the brakes are used, $\dot v = -b$ or $v = 0 \wedge \dot v = 0$.
Let $\varepsilon$ be an upper bound for the control loop time interval and $D$ a minimal distance we want to keep from the obstacle.
Then, as said in \cite{Mitsch-RSS-13}, for a velocity $v$, a distance less than $\frac {v^2} {2b} + \left(\frac A b + 1 \right) \left( \frac A 2 \varepsilon^2 + \varepsilon.v\right) + D$ should cause the vehicle to brake.

Like in previous section, we use $\varepsilon = MA(Input) + MA(Output) + MaxT(A)$.

\subsection{Correctness proof}

Because of the sensor accuracy $\epsilon_d$ on the distance and $\epsilon_v$ on the velocity, the controller sends the \emph{emergency stop} command as soon as the measured distance $d_m$ and velocity $v_m$ verify
\[ d_m - \epsilon_d \leq \frac {(v_m + \epsilon_v)^2} {2b} + \left(\frac A b + 1 \right) \left( \frac A 2 \varepsilon^2 + \varepsilon.(v_m + \epsilon_v)\right) + D\]
This way, if the actual distance is less than $\frac {v^2} {2b} + \left(\frac A b + 1 \right) \left( \frac A 2 \varepsilon^2 + \varepsilon.v\right) + D$, this command is sent even with sensor inaccuracy.

If the latest message available to the actuator contains the \emph{emergency stop} flag, it activates the brakes. Otherwise, it keeps a standard behavior (keep the same speed, respond to operator commands~\dots).
This allows for example to activate the brakes, and release them at the next step because the speed decreased, or because of different sensor errors (first, the distance was underestimated and the speed overestimated, then the contrary happened).

\begin{thm}\Thmname{Obstacle avoidance}

Assume at time $t$, the vehicle is in a safe situation: $d(t) \geq \frac {v(t)^2} {2b} + \left(\frac A b + 1 \right) \left( \frac A 2 \varepsilon^2 + \varepsilon.v(t)\right) + D$.
Then the vehicle stops with a minimum distance to the obstacle: $\forall s, \, d(s) \geq D$
\end{thm}

\begin{proof}
For simpler notation, let $\delta(v) = \frac {v(t)^2} {2b} + \left(\frac A b + 1 \right) \left( \frac A 2 \varepsilon^2 + \varepsilon.v(t)\right) + D$ and assume $d(s) < \delta(v(s))$ (the result is already proved otherwise). Let $l = lub(\{r \in [t, s], \, d(r) \geq \delta(v(r))\})$.

We have $d(l) = \delta(v(l)) \geq D + \frac A 2 \varepsilon^2 + \varepsilon.v(l)$, which is the maximum distance the vehicle can travel during $\varepsilon$. Then, if $s \leq l + \varepsilon$, the result is also proved.
Otherwise, with the same argument as in the proof of Theorem~\ref{control}, the brakes are always activated in $[l + \varepsilon, s]$. During this time, the vehicle travels at most (if the vehicle has not come to full stop yet) the distance needed to stop completely, which is 
\[\frac {v(l + \varepsilon)^2} {2b} \leq \frac{(v(l) + A.\varepsilon)^2}{2b}\]
\end{proof}


\bibliographystyle{plain}
\bibliography{biblio}


\end{document}
