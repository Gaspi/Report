\documentclass[12pt,a4paper,titlepage]{article}
\usepackage[utf8]{inputenc}
% \usepackage[francais]{babel}

\usepackage[left=1.5cm,right=2cm,top=2cm,bottom=2cm]{geometry}
\usepackage{amsmath}
\usepackage{amssymb}
\usepackage{amsthm}
\usepackage{amsfonts}
\usepackage{hyperref}
\usepackage{graphicx}

% --- CSL part
\usepackage{includes/natbib}
\usepackage{makebnf}

\usepackage[space]{grffile} % Celui-là je sais plus à quoi il sert...

% --- Partie code
\usepackage{listings}
\lstset{language=Java} % Changer eventuellement le nom du langage
\newcommand{\class}[1]{\texttt{#1}}

% --- Parti math. (app.)
\newtheorem{theorem}{Theorem}
\newtheorem{lemma}[theorem]{Lemma}
\newtheorem{proposition}[theorem]{Proposition}
\newtheorem{corollary}[theorem]{Corollary}
\newtheorem{definition}[theorem]{Definition}
\newtheorem{algorithm}[theorem]{Algorithm}
\newtheorem{remark}[theorem]{Remark}

% Mathematic abbreviations

\newcommand{\N}{\mathbb{N}}
\newcommand{\Z}{\mathbb{Z}}
\newcommand{\R}{\mathbb{R}}
\newcommand{\Esp}{\mathbb{E}}
%\newcommand{\E}{\mathbb{E}} % Si tu l'utilise beaucoup
\newcommand{\Prob}{\mathbb{P}}
%\newcommand{\P}{\mathbb{P}} % Si tu l'utilise beaucoup



\title{
\vspace{-3cm}
\normalsize
\begin{tabular}{p{15cm}}
ÉCOLE POLYTECHNIQUE \\
PROMOTION X2011 \\
Férey Gaspard
\end{tabular}
\vspace{3cm}
\begin{center}
\includegraphics[height=3cm]{pictures/logo.png}
\end{center}
\vspace{1cm}
\large
\begin{center}
RAPPORT DE STAGE D'OPTION SCIENTIFIQUE\\
\vspace{1cm}
{\Huge Titre}\\
\vspace{1cm}
NON CONFIDENTIEL
\end{center}
\vspace{3cm}
\normalsize
\begin{tabular}{p{6cm} p{10cm}}
Option :                  & INFORMATIQUE \\
Champ de l'option :       & Math-Informatique \\
Directeur de l'option :   & Olivier Bournez \\
Directeur de stage :      & Olivier Bournez \\
Dates du stage :          & 7 avril - 22 aout 2014\\
Nom et adresse de l'organisme :  & SRI International \\
                          & Computer Science Laboratory (CSL) \\
                          & 333 Ravenswood Avenue \\
                          & Menlo Park, CA 94025-3493 \\
                          & United States
\end{tabular}
}

\begin{document}

\maketitle
\tableofcontents
\newpage

\section{Introduction}

\section{PVS}


\section{Translating PVS to C}




\section{Types}
A PVS theory can be typechecked using the emacs interface \class{M-x typecheck} or with Lisp function \class{(tc name-theory)}. This first runs the PVS parser on the code and generates CLOS objects to represent it. Then, the PVS typechecker is run on this internal representation of the theory and tries to give a type to all expressions generating TCC when needed.\\

Here we describe how PVS types are represented in Lisp.




boolean, number, number\_field, real, 
rational, integer, $A \rightarrow B$, restricted types
$\text{below}(10) := \{ x:\text{int} | 0 \leq x < 10 \} $)  \\
enum
datatype

C types:
[unsigned] char, int, long, double
boolean
arrays
strings
enum
struct
and others: short int, float, union, size\_t, ...



\subsection{Fragment of PVS syntax}





\subsection{Difficulties}
if-expr
update-expr





\section{Other activities at SRI}

Robin project, HACMS \\
Contest week-end 14-15 June \\
Summer School \\






\section{Other works at SRI}
Discovering PVS :
Translating Coq proofs to PVS
PVS library for basic linear algebra


HACMS with Robin
Parsing Lisp code -> generate HTML architecture file
Correcting translator PVS to SMT-LIB

Summer school

\input{includes/bnf-theory}


\section{References}
Try using bibtex here

-- PVS --

PVS API Reference
PVS Lisp sources (github rep)
PVS language Reference
PVS System Guide
PVS Prelude library

-- Lisp --
Common Lisp Guy L. Steele Jr

-- C-- 
The C Library Reference Guide, Eric Huss

-- Other --
Compilation course, J.C. Filliatre


\bibliographystyle{te}
\bibliography{Report}


\end{document}
