



\begin{class}{expr}{abstract}{\classhln{syntax}}
{\slotidx{type} & the type of the expression}
\end{class}

\begin{class}{name}{mixin}{\classhln{syntax}}
{\slotidx{id} & the identifier \\
 \slotidx{actuals} & a list of actual parameters \\
 \slotidx{resolutions} & singleton}
This is a mixin for names, i.e., \texttt{name-expr}s, \texttt{type-name}s, etc.
\end{class}

\begin{class}{name-expr}{}{\classhln{name} \classhln{expr}}{}
\end{class}

\begin{class}{number-expr}{}{\classhln{expr}}{}
\end{class}

\begin{class}{tuple-expr}{}{\classhln{expr}}{\slotidx{exprs} & a list of expressions}
\end{class}

\begin{class}{application}{}{\classhln{expr}}%
{\slotidx{operator} & an expr \\
 \slotidx{argument} & an expr (maybe a tuple-expr)}
\end{class}

\begin{class}{field-application}{}{\classhln{expr}}%
{\slotidx{id} & identifier \\
 \slotidx{argument} & the argument}
A field application is the internal representation for record extraction,
e.g., \texttt{r`a}
\end{class}

\begin{class}{lambda-expr}{}{\classhln{binding-expr}}{}
This is the subclass of \texttt{binding-expr} used for LAMBDA expressions.
\end{class}

\begin{class}{if-expr}{}{\classhln{application}}{}
\end{class}

\begin{class}{record-expr}{}{\classhln{expr}}{\slotidx{assignments} & a list of assignments}
\end{class}

\begin{class}{update-expr}{}{\classhln{expr}}
{\slotidx{expression} & an expr \\
 \slotidx{assignments} & a list of assignments}
An update expression of the form \texttt{e WITH [x := 1, y := 2]}, maps to
an \texttt{update-expr} instance, where the \texttt{expression} is
\texttt{e}, and the \texttt{assignments} slot is set to the list of
generated \texttt{assigment} instances.
\end{class}

\begin{class}{assignment}{}{\classhln{syntax}}
{\slotidx{arguments} & the list of arguments \\
 \slotidx{expression} & the value expression}
Assignments occur in both record-exprs and update-exprs.\\
The \texttt{arguments} form is a list of lists.  For example, given the assignment \texttt{`a(x, y)`1 := 0}, the \texttt{arguments} are
\texttt{((a) (x y) (1))} and the \texttt{expression} is \texttt{0}.
\end{class}
