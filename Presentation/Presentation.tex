\documentclass{beamer}
\usepackage[utf8]{inputenc}


\usepackage{amsmath}
\usepackage{amssymb}
\usepackage{amsthm}
\usepackage{amsfonts}
% \usepackage{xcolor}
\usepackage{graphicx}


\usepackage[space]{grffile}


% --- Partie code
\usepackage{listings}
\lstset{language=C,
        showstringspaces=false,
        basicstyle=\footnotesize\ttfamily,
        captionpos=b,
        stepnumber=1,
        keywordstyle=\bfseries\color{green!40!black},
        commentstyle=\itshape\color{purple!40!black},
        identifierstyle=\color{blue},
        stringstyle=\color{red}}

\newcommand{\cl}[1]{\texttt{#1}}

% --- Parti math. (app.)
%\newtheorem{theorem}{Theorem}
%\newtheorem{lemma}[theorem]{Lemma}
%\newtheorem{proposition}[theorem]{Proposition}
%\newtheorem{corollary}[theorem]{Corollary}
%\newtheorem{definition}[theorem]{Definition}
%\newtheorem{algorithm}[theorem]{Algorithm}
%\newtheorem{remark}[theorem]{Remark}

% Mathematic abbreviations
\newcommand{\N}{\mathbb{N}}
\newcommand{\Z}{\mathbb{Z}}
\newcommand{\Q}{\mathbb{Q}}
\newcommand{\R}{\mathbb{R}}

\newcommand{\mpzt}{ \texttt{ mpz\_t } }
\newcommand{\mpqt}{ \texttt{ mpq\_t } }

\newcommand{\mut}{  \textbf{ mutable } }
\newcommand{\nmut}{ \textbf{ non-mutable } }
\newcommand{\bang}{ \textbf{ mutable } }
\newcommand{\safe}{ \textbf{ safe } }
\newcommand{\dupl}{ \textbf{ duplicated } }

% evaluation context hole
\newcommand{\econt}[1]{[#1]}
% update context hole
\newcommand{\ucont}[1]{\{#1\}}



\usetheme{Boadilla}
\usecolortheme{beaver}


\title[From PVS to C]{Translating PVS to C}
\subtitle{SRI International}
\author[Gaspard Férey]{Gaspard Férey}
\institute{Ecole Polytechnique}
\date{September 1st, 2014}


\begin{document}

\frame{\titlepage}

\begin{frame}
\frametitle{Table of Contents}
\tableofcontents
% \tableofcontents[currentsection]
\end{frame}


\section{Context}

\begin{frame}
\frametitle{What is PVS ?}

\end{frame}

\begin{frame}
\frametitle{Why translate PVS ?}

\begin{itemize}
\item To be able to execute PVS
\begin{itemize}
\item Testing
\item Debugging
\end{itemize}
\item To integrate PVS code into systems
\end{itemize}
\end{frame}




\section{Translating PVS}


\subsection{Architecture}

\begin{frame}
\frametitle{Translation steps}

\end{frame}

\subsection{Examples}

\begin{frame}
\frametitle{}

\end{frame}



\section{The update issue}

\subsection{Reference counting}
\begin{frame}
\frametitle{A reference counting GC}

\end{frame}


\subsection{Static analysis}
\begin{frame}
\frametitle{Semantics}

\end{frame}


\subsection{Algorithm}
\begin{frame}
\frametitle{The flags}

\end{frame}

\begin{frame}
\frametitle{Algorithm}

\end{frame}


\section{Conclusion}

\begin{frame}
\frametitle{Conclusion}

\end{frame}



\begin{frame}
\frametitle{Questions ?}
\begin{center}
\includegraphics[scale=0.5]{includes/questions.jpg}
\end{center}
\end{frame}


\subsection{Demonstration}

\begin{frame}
\frametitle{Demonstration}
\begin{center}
\includegraphics[scale=1.5]{includes/demogods.jpg}
\end{center}
\end{frame}


\begin{frame}
\frametitle{Questions ?}
\begin{center}
\includegraphics[scale=0.5]{includes/questions.jpg}
\end{center}
\end{frame}




\end{document}
